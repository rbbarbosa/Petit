Restriction: there is a single symbol table for the entire program. As a consequence, functions and parameters cannot share the same name (even parameters of different functions).
\begin{enumerate}
\item Modify the code so that undeclared symbols are reported through an error message.
\item Modify the code to show the line and column numbers where semantic errors are identified. There are several possibilities: lexical analysis returns a token object; use locations.
\item Modify the code so that \textit{expression} nodes of categories Identifier, Natural and Decimal are annotated with the type (integer\_type or double\_type). Then implement the following semantic check: assigning a double (Identifier or Decimal) to an integer variable should report a compiler warning: implicit conversion from double to integer.
\item Final, long exercise: check that variables used in expressions are parameters of the respective function. This requires scopes...
\end{enumerate}

% check only for the right number of arguments! (not their type, ever, because in Petit we can promote and underpromote)

% each scope (loop, if) could have its own variables and there could be an exercise on that

% Exercise: show the symbol table

% Exercise: show tree with type annotations

% trivial exercise: check for existence of main function with integer parameter (maybe we don't want this, actually, to generate basic code without a main function, at the beginning)